\begin{lstlisting}
%% set paths

TRENTOOLpath = './TRENTOOL3';
addpath(TRENTOOLpath); 
addpath('./fieldtrip-20150928');
ft_defaults;

%% get data

% find relevant mat files and collect file names in a cell structure
fileCell = {'subj_A_cond_1.mat' 'subj_A_cond_2.mat'};

%% create cfg (as you would for single subject analysis)

cfgTEGP                  = [];
cfgTEGP.TEcalctype       = 'VW_ds'; 

% get channel label from file, analyze all possible combinations
load(files(1).name);
cfgTEP.channel             = data.label;

% specify u to be scanned
cfgTEGP.predicttimemin_u    = 5;
cfgTEGP.predicttimemax_u    = 17;
cfgTEGP.predicttimestepsize = 2;

% use the whole trial for analysis
cfgTEP.toi = [min(data.time{1,1}),max(data.time{1,1})]; % time of interest

% ACT estimation and constraints on allowed ACT(autocorelation time)
cfgTEGP.maxlag      = 3*cfgTEGP.predicttimemax_u;   % range for ACT computation
cfgTEGP.actthrvalue = 40;                           % treshold for ACT 
cfgTEGP.minnrtrials = 12;                           % minimum acceptable number of trials

% optimizing embedding
cfgTEGP.optimizemethod ='ragwitz';   % criterion used
cfgTEGP.ragdim         = 4:8;        % criterion dimension
cfgTEGP.ragtaurange    = [0.2 0.4];  % range for tau [0.5 1]
cfgTEGP.ragtausteps    = 15;         % steps for ragwitz tau steps
cfgTEGP.repPred        = 100;        % points used for optimization the embedding dimensions

% kernel-based TE estimation
cfgTEGP.flagNei = 'Mass' ;           % neigbour analyse type
cfgTEGP.sizeNei = 4;                 % neigbours to analyse

cfgTEGP.ensemblemethod  = 'no';
cfgTEGP.outputpath  = '/results/';

%% run TEgroup_prepare

TEgroup_prepare(cfgTEGP,fileCell);

%% run analysis on group prepared data

cfgTESS		      = [];
cfgTESS.shifttest     = 'yes';
cfgTESS.optdimusage   = 'indivdim';   % dimension to use
cfgTESS.surrogatetype = 'trialshuffling';

% go to output path of previous analysis step (this is where TEgroup_prepare saved the prepared data)
cd(cfgTEGP.outputpath)

% load and analyze prepared data individually
for i=1:length(fileCell)
        
    load(fileCell(i).name)
    cfgTESS.fileidout = strcat(cfgTEGP.outputpath,fileCell(i).name(1:end-4));
    
    TEresult = InteractionDelayReconstruction_calculate(cfgTEGP,cfgTESS,data);
    
    save([cfgTEGP.outputpath fileCell(i).name(1:end-4) '_calc.mat'],'TEresult');
    clear TEresult data;
    
end

%% perform graph analysis (optional)

files = dir('*_calc.mat');


% settings for graph analysis
cfgGA = [];
cfgGA.threshold = 1;
cfgGA.cmc       = 1;

for i=1:length(files)
    
    load(files(i).name);
    
    % correct for spurious information transfer
    TEresult_GA = TEgraphanalysis(cfgGA,TEresult);
    
    save([cfgTEGP.outputpath files(i).name(1:end-9) '_result.mat'],'TEresult_GA');
    
end


%% run compare TE values from both conditions

% collect individually analyzed data in a file cell
cd(cfgTEGP.outputpath);
files_TEpermtest = dir('*_result.mat');
cond1 = load(files_TEpermtest(1).name);
cond2 = load(files_TEpermtest(2).name);



% define analysis parameters
cfgCSTAT = [];
cfgCSTAT.permstatstype = 'indepsamplesT';
cfgCSTAT.alpha 	       = 0.05;
cfgCSTAT.tail 	       = 1;
cfgCSTAT.fileidout = 'test_condstatssingle';

% run statistics
TEgroup_conditionstatssingle(cfgCSTAT, cond1.TEpermtest, cond2.TEpermtest);
\end{lstlisting}