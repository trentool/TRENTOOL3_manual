% layout for MATLAB listings used in the TRENTOOL documentation

% define colors used in listings
\definecolor{lightgray}{RGB}{230,230,230}
\definecolor{lightyellow}{RGB}{252,245,224}
\definecolor{keyblue}{RGB}{20,7,252}
\definecolor{commentgreen}{RGB}{83,133,50}
\definecolor{stringpink}{RGB}{197,73,205}

% FONTS:
%\textrm{..}	 \rmfamily	 R�misch
%\textit{..}	 \itshape	 Schr�ger Text (auch \emph{})
%\textbf{..}	 \bfseries	 Fetter Text
%\textsc{..}	 \scshape	 Text in Kapit�lchen
%\texttt{..}	 \ttfamily	 Schreibmaschinentext
%\textnormal{..}	 \normalfont	 Standardfont im Dokument

\lstset{
basicstyle=\footnotesize\ttfamily, 
keywordstyle=\bfseries\color{keyblue},
commentstyle=\color{commentgreen},
stringstyle=\color{stringpink},
numbers=left,                   % where to put the line-numbers
numberstyle=\tiny,      % the size of the fonts that are used for the line-numbers
stepnumber=1,                   % the step between two line-numbers. If it's 1, each line 
numbersep=5pt,                  % how far the line-numbers are from the code
backgroundcolor=\color{lightyellow},  % choose the background color. You must add \usepackage{color}
showspaces=false,
breaklines=true,
}

%\lstset{ %
%language=Matlab,                % the language of the code
%basicstyle=\footnotesize       % the size of the fonts that are used for the code
%basicstyle=\scriptsize\ttfamily,

%keywordstyle=\bfseries,\color{blue},
%identifierstyle=,
%commentstyle=\color{lightgray},	
%stringstyle=\itshape\color{magenta},
%
%numbers=left,                   % where to put the line-numbers
%numberstyle=\tiny,      % the size of the fonts that are used for the line-numbers
%stepnumber=1,                   % the step between two line-numbers. If it's 1, each line 
%                                % will be numbered
%numbersep=5pt,                  % how far the line-numbers are from the code
%backgroundcolor=\color{lightgray},  % choose the background color. You must add \usepackage{color}
%showspaces=false,               % show spaces adding particular underscores
%showstringspaces=false,         % underline spaces within strings
%showtabs=false,                 % show tabs within strings adding particular underscores
%%frame=single,                   % adds a frame around the code
%tabsize=2,                      % sets default tabsize to 2 spaces
%captionpos=t,                   % sets the caption-position to bottom
%breaklines=true,                % sets automatic line breaking
%breakatwhitespace=false,        % sets if automatic breaks should only happen at whitespace
%%title=\lstname,                 % show the filename of files included with \lstinputlisting;
%                                % also try caption instead of title
%escapeinside={\%*}{*)},         % if you want to add a comment within your code
%morekeywords={*,...},            % if you want to add more keywords to the set
%float=H,
%}