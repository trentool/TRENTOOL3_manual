\begin{table}[H]
\small \centering
\caption[Parameters \texttt{cfgBN}]{Parameters for the configuration structure \texttt{cfgBN} of the function \texttt{TRENTOOL2BrainNet.m} (TRENTOOL Version 3.3).} 
\begin{tabularx}{\textwidth}{p{2cm}L{2.5cm}X} \toprule
\textbf{field name} & \textbf{data type} & \textbf{description} \\ \midrule
\verb+MNIcoord+ & integer array [nx3] & MNI coordinates (x,y,z) of $n$ sources or channels.\\
\rowcolor{Gray}
\verb+labels+ & cell array [nx1] & Cell array containing strings with source/channel labels. \textbf{NOTE}: Labels should not contain blanks, this causes BrainNet Viewer to chrash. Also, make sure to use the same ordering of labels for the fields \texttt{.labels} and \texttt{.MNIcoord} and the TEpermvalues table in \texttt{TEpermtest} (you may use the label information in \texttt{TEpermtest.cfg.channel}).\\
\verb+filename+ & string & filename for output files \texttt{*.node} and \texttt{*.edge}, you may specify a filename only (string), which causes the function to save both files to the current folder, or you can specify a whole filepath ([filepath]/[filename]).\\
\rowcolor{Gray}
\verb+nodeCol+ & integer vector [nx1] & Optional. BrainNet gives you the option to color nodes according to the values in this vector, see BrainNet Manual. If you want the nodes to have the same color, provide a vector with ones or no vector at all.\\ 
\verb+nodeSize+ & integer vector [nx1] & Optional. BrainNet gives you the option to size nodes according to the values in this vector, see BrainNet Manual. If you want the nodes to have the same size, provide a vector with ones or no vector at all.\\ 
\rowcolor{Gray}
\verb+edgeCol+ & integer vector [nx1] & Optional. BrainNet gives you the option to color edges according to the values in this vector, see BrainNet Manual. If you want the edges to have the same color, provide a vector with ones or no vector at all.\\ \bottomrule
\end{tabularx} \label{tab:cfgBN}
\end{table}

% * INPUT
%   TEpermtest = structure containing TE results, returned by TRENTOOL
%                functions TEsurrogatestats.m, TEsurrogatestats_ensemble.m 
%                or InteractionDelayReconstruction_analyze.m
%
%  cfg. 	 Configuration structure with fields
%   MNIcoord   = MNI coordinates (x,y,z) of the sources, array with size 
%                [N 3], where N is the number of sources
%   labels     = source labels, cell array with size [N 1] - Not that
%                sources shouldn't contain spaces (causes BrainNet to
%                crash), the function will remove any spaces from the 
%                labels cell array. WARNING: You may use the labels
%                provided in TEpermtest.cfg.channel, if you use different
%                labels, make sure, they are in the same order as the
%                channels in TEpermtest.cfg.channel!
%   filename   = filename for output files *.node and *.edge, you may
%                specify a filename only (string), which causes the 
%                function to save both files to the current folder, or you 
%                can specify a whole filepath ([filepath]/[filename])
%
%
%   THE FOLLOWING INPUTS ARE OPTIONAL:
%
%  cfg.
%   nodeCol    = BrainNet gives you the option to color nodes according to
%                the values in this vector (size [N 1]), see BrainNet
%                Manual. If you want the nodes to have the same color,
%                provide a vector with ones (default).
%   nodeSize   = BrainNet gives you the option to size nodes according to
%                the values in this vector (size [N 1]), see BrainNet
%                Manual. If you want the nodes to have the same size,
%                provide a vector with ones (default).
%   edgeCol    = BrainNet gives you the option to color edges according to
%                the values in this vector (size [N 1]), see BrainNet
%                Manual. If you want the edges to have the same color,
%                provide a vector with ones (default).