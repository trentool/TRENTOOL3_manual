\begin{table}[H]
\small
\caption[Parameters \texttt{cfgTEA}]{Parameters for the configuration structure \texttt{cfgTEA} of the function \texttt{InteractionDelayReconstruction\_analyze.m} (all parameters are given as strings) (TRENTOOL Version 3.0)} 
\begin{tabularx}{\textwidth}{lX} \toprule
\textbf{field name} & \textbf{description} \\ \midrule
\verb+select_opt_u+ & selects the way the optimal u is determined options are: \\
                 &        \verb+'min_p'+ - optimal predictiontime u is the one
                         with the largest statistical distance (smallest
                         randomization p-value) to  surrogate data.
                         This option might be problematic with
                         respect to later testing of existence of
                         a link if not used on independent data first. \\

                 &        \verb+'max_TEdiff'+ - optimal predictiontime u is the
                         one with the largest difference in the test
                         statistic between data and surrogates.
                         This option might be problematic if different
                         predictiontimes u lead to vastly different
                         embedding via the optimization in the ragwitz
                         criterion. \\
                                                                              
                 &       \verb+'product_evidence'+ - optimal predictiontime u is
                        the one which maximes the product (1-p)*TEdiff.
                        is is a statistically weighted measure of
                        TEdifferences between data and surrogates
                        (experimental feature)\\
\rowcolor{Gray}
\verb+select_opt_u_pos+ & 'shortest' select the shortest u if multiple u's optimize the target quantity (minimum p, maximum TE difference); 'longest' select the longest u that optimizes the target quantity. \\
\bottomrule
\label{tab:cfgTEA}
\end{tabularx}
\end{table}